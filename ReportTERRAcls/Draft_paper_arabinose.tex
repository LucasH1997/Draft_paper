%%%%%%%%%%%%%%%%%%%%%%%%%%%%%%%%%%%%%%%%%%%%%%%%%%%%%%%%%%%%%%%%%%%%
% This is just an example/guide for you to refer
% This template and example was modified from the FrontiersSCN.cls 
% It is only intended for internal use
%%%%%%%%%%%%%%%%%%%%%%%%%%%%%%%%%%%%%%%%%%%%%%%%%%%%%%%%%%%%%%%%%%%

%%% Version 1.0 Generated 2020/03/06 %%% 
%%% Contact ja.martinez@uliege.be
%%% You will need to have the following packages installed: changes,datetime, fmtcount, etoolbox, fcprefix, which are normally inlcuded in WinEdt. %%%
%%% Also frontiersinSCNS_ENG_HUMS.bst is recommended for bibliography %%%
%%% In http://www.ctan.org/ you can find the packages and how to install them, if necessary. %%%
%%%  GABTlogo.png is required in the path in order to correctly compile front page header %%%

\documentclass[utf8]{ReportTERRA} 
%\setcitestyle{square} % if preferred
\usepackage{url,hyperref,lineno,microtype,subcaption}
\usepackage[onehalfspacing]{setspace}
\usepackage{lipsum}  %ignore this package in your own manuscript, it is intended to print dummy text

\linenumbers
% define the names and colors of Track changes for people revising the document for any new add a new line
\definechangesauthor[name={auth1}, color=orange]{auth1}
\definechangesauthor[name={auth2}, color=red]{auth2}

% Leave a blank line between paragraphs instead of using \\ if using \\ note that line number will also add 1 line. 

\def\keyFont{\fontsize{8}{11}\helveticabold }
\def\firstAuthorLast{Henrion {et~al.}} %use et al only if is more than 1 author
\def\Authors{Lucas Henrion\,$^{1,*}$, Juan Andreas Martinez\,$^{2}$ and Frank Delvigne\,$^{1,2}$}
% Affiliations should be keyed to the author's name with superscript numbers and be listed as follows: Laboratory, Department).

\def\Address{$^{1}$Liège University GxABT, MIPI,\\ % \\ separation
$^{2}$Liège University GxABT, MIPI }
% The Corresponding Author Full name or Initials for this comunication should be marked with an asterisk
\def\corrEmail{lucas.henrion@uliege.be}


\begin{document}
\onecolumn
\firstpage{1}

\title[Running Title]{Draft: What is hiding behind phenotypic diversity in chemostat mode cultivation? } 

\author[\firstAuthorLast ]{\Authors} %This field will be automatically populated
\address{} %This field will be automatically populated
\correspondance{} %This field will be automatically populated

\extraAuth{}% If there are more than 1 corresponding author, comment this line and uncomment the next one.
%\extraAuth{corresponding Author2 \\ email2@uni2.edu}

\maketitle

\begin{abstract}
This is where an abstract can be written, if no need please ignore this section. Abstract can contain individual sections as follows.
\section{Abstract section 1:}
\lipsum[1]
\section{Abstract section 2:}
\lipsum[2]
\tiny
 \keyFont{ \section{Keywords:} keyword, keyword, keyword, keyword, keyword, keyword, keyword, keyword} %All article types: you may provide up to 8 keywords; at least 5 are mandatory.
\end{abstract}

\newpage %this should be commented if no abstract or small abstract or at will.

\section{Introduction}

\begin{itemize}
\item Phenotypic diversity is viewed as the main contributor to population heterogeneity during the first 100 hours of cultivation
\item Chemostat cultivation, witch as all continuous process represents a improvement compare to traditional batch and fed-batch, as been described as a highly competitive environment thus exercising a selection pressure and a stress on the bacterial population
\item The popular and well studied arabinose operon has been shown to experience of extreme form of phenotypic heterogeneity named bimodality in such cultivation mode
\item In a previous paper from our team we describe how such heterogeneity could me mitigated by adding arabinose not in a constant feed but as a pulse
\item By describing how we can control the population with the segregostat we uncover a key contributor to phenotypic heterogeneity in continuous cultivation

\end{itemize}

\section{Key papers}

In this section I describe global ideas from key papers that I want to cite.

\subsection{Global concept 1: single cell analysis as a reseach and control tool}



\subsection{Global concept 2: Phenotypic diversity is a multicomponant process that remains poorly caracterise}

\subsection{Global concept 3: Gene activation is comprised of multiple input that interact as a gene regulation network}

\section{Observation}

Arabinose both serves as an inductor and a carbon source. Single cell induction has a hill shape probability density that depends on the inductor concentration. Induction leads to the consumption of arabinose towards an equilibration point. Because the system has memory, the primary induction overshoot that stable point and leads to a de-induction of the system. Concequantly the arabinose concentration rises again giving bith to a secondary induction. The system then oscillates towards that equilibration point and bimodality.

\section{Model description}

We have built a mechanistic model to explain and simulate the population induction profile. The population is split into 2000 imaginary cells that can each interact with the substrate in a random order. A group can either  only consume glucose and ignore arabinose or if it had invested in the transportors and enzymes, gain the additional capacity to consume arabinose. The choice of each group to co-consume or not depends on the induction of the arabinose operon given a Hill shape density probability. Once a bacterial groupe has decided to commit to arabinose it will only start co-consuming after a delay. This last represents the time needed for the transcription and translation of the operon. Beside, if a bacteria commits to co-consumption it will remain capable of doing so until it devides, even if the arabinose content has fell bellow induction limit.

\subsection{Equations}
Equations should be inserted in editable format from the equation editor, align can also be used.

\begin{equation}
\sum x+ y =Z\label{eq:01}
\end{equation}

example of align.

\begin{align}
\sum (x) + y &= Z\label{eq:02}\\
\sum (x) &= Z - y\label{eq:03}
\end{align}




\section{Track Changes}
%%%note that by puting [final] as option on document class all track changes and remarks will disapear and final document will be compiled.

%See that remarks are optional
This is \added[id=auth1,remark={we need this}]{new} text.
This is \added[id=auth2]{new} text.
This is \deleted[id=auth1,remark=obsolete]{unnecessary}text.
This is \replaced[id=auth1]{nice}{bad} text.

% can also be performed without the need of defined author
This is \added[remark={we need this}]{new} text.
This is \added[remark={has to be in it}]{new} text.
This is \deleted[]{unnecessary}text.
This is \replaced{nice}{bad} text.

% A list of changes can be made by the following command
\listofchanges

\bibliographystyle{frontiersinSCNS_ENG_HUMS} 
\bibliography{test}


\section*{Figures}

\begin{figure}[h!]
\begin{center}
\includegraphics[width=10cm]{GABTlogo.png}
\end{center}
\caption{Enter the caption for your figure here.  Repeat as  necessary for each of your figures}\label{fig:1}
\end{figure}


\end{document}
