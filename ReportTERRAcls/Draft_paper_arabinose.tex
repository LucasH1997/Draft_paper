%%%%%%%%%%%%%%%%%%%%%%%%%%%%%%%%%%%%%%%%%%%%%%%%%%%%%%%%%%%%%%%%%%%%
% This is just an example/guide for you to refer
% This template and example was modified from the FrontiersSCN.cls 
% It is only intended for internal use
%%%%%%%%%%%%%%%%%%%%%%%%%%%%%%%%%%%%%%%%%%%%%%%%%%%%%%%%%%%%%%%%%%%

%%% Version 1.0 Generated 2020/03/06 %%% 
%%% Contact ja.martinez@uliege.be
%%% You will need to have the following packages installed: changes,datetime, fmtcount, etoolbox, fcprefix, which are normally inlcuded in WinEdt. %%%
%%% Also frontiersinSCNS_ENG_HUMS.bst is recommended for bibliography %%%
%%% In http://www.ctan.org/ you can find the packages and how to install them, if necessary. %%%
%%%  GABTlogo.png is required in the path in order to correctly compile front page header %%%

\documentclass[utf8]{ReportTERRA} 
%\setcitestyle{square} % if preferred
\usepackage{url,hyperref,lineno,microtype,subcaption}
\usepackage[onehalfspacing]{setspace}
\usepackage{lipsum}  %ignore this package in your own manuscript, it is intended to print dummy text

\linenumbers
% define the names and colors of Track changes for people revising the document for any new add a new line
\definechangesauthor[name={auth1}, color=orange]{auth1}
\definechangesauthor[name={auth2}, color=red]{auth2}

% Leave a blank line between paragraphs instead of using \\ if using \\ note that line number will also add 1 line. 

\def\keyFont{\fontsize{8}{11}\helveticabold }
\def\firstAuthorLast{Sample {et~al.}} %use et al only if is more than 1 author
\def\Authors{Lucas Henrion\,$^{1,*}$, Juan Andreas Martinez\,$^{2}$ and Frank Delvigne\,$^{1,2}$}
% Affiliations should be keyed to the author's name with superscript numbers and be listed as follows: Laboratory, Department).

\def\Address{$^{1}$Liège University GxABT, MIPI,\\ % \\ separation
$^{2}$Liège University GxABT, MIPI }
% The Corresponding Author Full name or Initials for this comunication should be marked with an asterisk
\def\corrAuthor{Corresponding Author}
\def\corrEmail{email@uni.edu}


\begin{document}
\onecolumn
\firstpage{1}

\title[Running Title]{Report Title} 

\author[\firstAuthorLast ]{\Authors} %This field will be automatically populated
\address{} %This field will be automatically populated
\correspondance{} %This field will be automatically populated

\extraAuth{}% If there are more than 1 corresponding author, comment this line and uncomment the next one.
%\extraAuth{corresponding Author2 \\ email2@uni2.edu}

\maketitle

\begin{abstract}
This is where an abstract can be written, if no need please ignore this section. Abstract can contain individual sections as follows.
\section{Abstract section 1:}
\lipsum[1]
\section{Abstract section 2:}
\lipsum[2]
\tiny
 \keyFont{ \section{Keywords:} keyword, keyword, keyword, keyword, keyword, keyword, keyword, keyword} %All article types: you may provide up to 8 keywords; at least 5 are mandatory.
\end{abstract}

\newpage %this should be commented if no abstract or small abstract or at will.

\section{Introduction}
\lipsum[3-4]

\section{Other section}
\lipsum[5-6]

\subsection{Heading Levels}

%There are 5 heading levels

\subsection{Level 2}
\subsubsection{Level 3}
\paragraph{Level 4}
\subparagraph{Level 5}

\subsection{Equations}
Equations should be inserted in editable format from the equation editor, align can also be used.

\begin{equation}
\sum x+ y =Z\label{eq:01}
\end{equation}

example of align.

\begin{align}
\sum (x) + y &= Z\label{eq:02}\\
\sum (x) &= Z - y\label{eq:03}
\end{align}



\section{Citation}

For Original Research Articles \citep{conference}, Clinical Trial Articles \citep{article}, and Technology Reports \citep{patent}, the introduction should be succinct, with no subheadings \citep{book}. For Case Reports the Introduction should include symptoms at presentation \citep{chapter}, physical exams and lab results \citep{dataset}.

\section{Track Changes}
%%%note that by puting [final] as option on document class all track changes and remarks will disapear and final document will be compiled.

%See that remarks are optional
This is \added[id=auth1,remark={we need this}]{new} text.
This is \added[id=auth2]{new} text.
This is \deleted[id=auth1,remark=obsolete]{unnecessary}text.
This is \replaced[id=auth1]{nice}{bad} text.

% can also be performed without the need of defined author
This is \added[remark={we need this}]{new} text.
This is \added[remark={has to be in it}]{new} text.
This is \deleted[]{unnecessary}text.
This is \replaced{nice}{bad} text.

% A list of changes can be made by the following command
\listofchanges

\bibliographystyle{frontiersinSCNS_ENG_HUMS} 
\bibliography{test}


\section*{Figures}

\begin{figure}[h!]
\begin{center}
\includegraphics[width=10cm]{GABTlogo.png}
\end{center}
\caption{Enter the caption for your figure here.  Repeat as  necessary for each of your figures}\label{fig:1}
\end{figure}


\end{document}
